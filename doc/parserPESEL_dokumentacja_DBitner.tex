\documentclass[12pt,a4paper]{article}
\usepackage[polish]{babel}
\usepackage[T1]{fontenc}
\usepackage[utf8x]{inputenc}
\usepackage{hyperref}
\usepackage{url}
\usepackage{graphicx}

\addtolength{\hoffset}{-1.5cm}
\addtolength{\marginparwidth}{-1.5cm}
\addtolength{\textwidth}{3cm}
\addtolength{\voffset}{-1cm}
\addtolength{\textheight}{2.5cm}
\setlength{\topmargin}{0cm}
\setlength{\headheight}{0cm}

\begin{document}
	
	\title{Dokumentacja projektu\\ Parser PESEL \\ Języki skryptowe}
	\author{Dawid Bitner, IA}
	\date{\today}
	
	\maketitle
	\begin{center}				
		\vspace{14cm} Politechnika Śląska \\ Wydział Matematyki Stosowanej \\ Rok akademicki 2018/2019
	\end{center}
	\newpage 
	
	\section*{Część I}
	\subsection*{Opis programu}
	Program ma za zadanie sprawdzanie poprawności podanego numeru \textbf{Powszechnego Elektronicznego System Ewidencji Ludności (PESEL)}. Program działa na zasadzie niewielkiej bazy danych przechowującej rekordy i potrafiącej tworzyć kopię zapasową wyników.
	\subsection*{Instrukcja obsługi}
	Aby dodać kolejne numery PESEL do bazy, w celu sprawdzenia ich poprawności należy tworzyć kolejne pliki \textit{inX.txt}, w folderze \textit{in} gdzie \textit{X} to kolejna liczba określająca numer pliku wejściowego. W celu uruchomienia programu należy uruchomić plik \textit{skrypt.bat}, wyniki działania programu zostaną zapisane w folderze \textit{out}, a raport wygenerowany do folderu \textit{backup}, oraz jeżeli użytkownik ma poprawnie skojarzone otwieranie plików \textit{HTML} z przeglądarką - otwarty w domyślnej przeglądarce.
	\begin{figure}[ht]
    	\centering
  		\includegraphics[width=1\textwidth]{1.png}    
    \end{figure}


\newpage
	\section*{Część II}
	\subsection*{Część techniczna}
	Program uruchamiamy poprzez normalne uruchomieniu pliku \textit{skrypt.bat}, lub wywołania go za pomocą linii komend. Skrypt konsolowy ma za zadanie wywoływanie kolejnych elementów programów tj.: \\ 1. Stworzenie folderów: \textit{in}, \textit{out} i \textit{backup} jeżeli nie istnieją. \\ 2. Wywołanie \textit{parserPESEL.exe} - algorytmu napisanego w języku \textit{C++} sprawdzającego poprawność numerów PESEL. \\3. Wywołanie skryptu \textit{skrypt\char`_parser.py} napisanego w języku \textit{Python} generującego raport/kopię zapasową.\\W celu uproszczenia działania programu powyższe pliki  zostały umieszczone w folderze nadrzędnym do folderów \textit{in}, \textit{out} i \textit{backup}.
	\subsection*{Opis działania} 
		\subsubsection*{skrypt.bat}
		Skrypt w pierwszej kolejności tworzy foldery: \textit{in}, \textit{out} i \textit{backup} jeżeli nie istnieją. Następnie wywołuje kolejno \textit{parserPESEL.exe} i \textit{skrypt\char`_python.py} informując o swoich działaniach użytkownika.
			\begin{figure}[ht]
    	\centering
  		\includegraphics{2.png}    
    \end{figure}
		
		\subsubsection*{parserPESEL.exe}
		Odpowiada za część algorytmiczną programu. Każdy wpis w rejestrze PESEL jest określany unikatowym symbolem jednoznacznie identyfikującym osobę fizyczną. Numer PESEL jest to 11-cyfrowy stały symbol numeryczny jednoznacznie identyfikujący określoną osobę fizyczną. Zbudowany jest z następujących elementów: zakodowanej daty urodzenia, liczby porządkowej, zakodowanej płci, cyfry kontrolnej.
\\		
Przykładowa postać:\\
440514 	0145 	8
\\
    cyfry [1-6] – data urodzenia z określeniem stulecia urodzenia\\
    cyfry [7-10] – numer serii z oznaczeniem płci\\
        cyfra [10] – płeć\\
    cyfra [11] – cyfra kontrolna	\\	
		\subsubsection*{skrypt\char`_python.py}
Odpowiada za stworzenie kopii zapasowej/raportu w postaci tabeli w pliku \textit{html}. Raport jest tworzony na zasadzie kopii przyrostowej tj. jeżeli istnieje kopia stworzona wcześniej, zostaje ona wczytana, zmienna sprawdzająca ilość plików wyjściowych zostaje ustawiona na liczbę plików + 1 i dopisywanie wyników rozpoczyna się od pierwszego nowego pliku który nie został zawarty w poprzednim raporcie. Jeżeli żadna kopia nie została wcześniej utworzona, skrypt wypisze do pliku \textit{html} wszystkie dane w plikach wejściowych i wyjściowych. Nazwa pliku raportu jest w formacie \textit{timestamp.html}, gdzie \textit{timestamp} to dokładny czas stworzenia raportu w postaci: \textit{ROK-MIESIAC-DZIEN\char`_GODZINA-MINUTA-SEKUNDA}.
	\subsection*{Implementacja}
	\subsubsection*{parserPESEL.exe}
Schemat działania:\\
-wyszukaj pliki \textit{out} z rozszerzeniem \textit{.txt} w folderze \textit{out}\\
-ostatni plik \textit{outX.txt} zostanie zapisany do zmiennej, jeżeli nie istnieje to zmienna wynosi 0
-wyciągamy \textit{X} z nazwy pliku, zapisujemy do zmiennej typu \textit{int}\\
-wczytujemy odpowiednie pliki \textit{inX.txt} z folderu \textit{in}, zaczynamy od \textit{X}, gdzie \textit{X} to \textit{X} z ostatniego pliku \textit{out} + 1\\
-dla każdego pliku wejściowego wykonujemy poniższy algorytm, a jego wynik zapisujemy do odpowiedniego pliku wyjściowego:\\
-algorytm wywołuje kolejne metody sprawdzające poprawność numeru PESEL, jeżeli w którymś miejscu walidator znajdzie nieprawidłowość zwróci wartość zmiennej typu \textit{bool} jako \textit{false}, w przeciwnym wypadku jako \textit{true}, operacja ta wykonywana jest dla każdego kolejnego pliku wejściowego\\
-przy wartości \textit{true} do pliku wyjściowego zostaje wypisana informacja o poprawności numeru, dacie urodzenia tej osoby i jej płci, w przeciwnym wypadku w pliku wyjściowym zostaje wypisana informacja o nieprawidłowości walidacji\\
-jeżeli nie istnieje już żaden plik wejściowy, kończymy działanie programu\\
	\subsubsection*{skrypt\char`_python.py}
Schemat działania:\\
-skrypt sprawdza czy w folderze \textit{backup}	istnieje jakaś kopia zapasowa, jeżeli tak zostanie skopiowana, a nowe dane będą dopisywane poniżej, jeżeli nie zostanie stworzony plik \textit{html} od podstaw, zawierający nagłówki i dołączenie arkusza stylów\\
-z ostatniego pliku wyjściowego zostaje wyciągnięta liczba, mówiąca o ilości plików, zostaje zwiększona o 1\\
-w postaci kopii przyrostowej zostają dopisywane kolejne komórki tabeli zawierające treść plików wejściowych i odpowiadających im plików wyjściowych\\

	\section*{Ogólny schemat blokowy}
	\begin{figure}[ht]
    	\centering
  		\includegraphics[width=0.75\textwidth]{diagram.png}    
    \end{figure}	

	\section*{Pełen kod programu}
Kod zawiera komentarze dokładnie tłumaczące działanie wybranych fragmentów programu.	
	\subsection*{skrypt.bat}
	
	\begin{verbatim}
	@echo off
IF NOT EXIST in mkdir in 
IF NOT EXIST out mkdir out
IF NOT EXIST backup mkdir backup
echo Trwa uruchamianie programu parserPESEL.exe
parserPESEL.exe
echo parserPESEL.exe zakonczyl prace
echo Trwa uruchamianie skrypt_parser.py
py -u skrypt_parser.py
echo skrypt_parser.py zakonczyl prace
echo otwieranie raportu
pause

	\end{verbatim}
	
	\subsection*{parserPESEL.exe}
	
	\subsubsection*{main.cpp}
	\begin{verbatim}
#include "pesel.h"

int main(){
    pesel Pesel;

    char PESEL[12] = "";	// prawidlowy numer PESEL ma dokladnie 11 cyfr, 
    wystarczy ze jest dluzszy o 1 cyfra zeby byl nieprawidlowy
    int i;

	/*rozwiazanie problemu za:
	https://stackoverflow.com/questions/26475540/c-having-problems-with-a-simple-example-of
	-findfirstfile */

    string DATA_DIR = "out/out*.*"; // wyszukaj tylko pliki out*.* w folderze out; 
    string lastFile = ""; // tworzenie stringu, rownie dobrze moze byc samo 
    string lastFile;

    HANDLE hFind; // uchwyt
    WIN32_FIND_DATAA data; // https://docs.microsoft.com/
    en-us/windows/desktop/api/minwinbase/ns-minwinbase-_win32_find_dataa

    hFind = FindFirstFileA(DATA_DIR.c_str(), &data); // https://docs.microsoft.com/
    en-us/windows/desktop/api/fileapi/nf-fileapi-findfirstfilea

    if(hFind != INVALID_HANDLE_VALUE){
        while(FindNextFile(hFind, &data)); // czytanie kazdego pliku, ostatni zostanie zapisany w data
        lastFile = data.cFileName; // lastFile to nazwa pliku
        FindClose(hFind); // zamknij handler
    }

	if(lastFile.length()){ // jesli to nie jest pusty string, czyli jesli istnieje juz 
	jakis out
		lastFile = lastFile.substr(3, lastFile.length() - 4 - 3); // 4 = ".txt"; 3 = "out"
		 i = atoi(lastFile.c_str()) + 1; // string -> char[] -> int
	}
	else // jesli jednak zaden out nie istnieje
		i = 1; // to zacznij od pierwszego

		cout << lastFile;

    do{
		stringstream ss; // z <sstream>, int to string
		ss << i; // z <sstream>, int to string

		//---OTWIERANIE PLIKU WEJSCIOWEGO---//
        string filenameInput = "in/in" + ss.str(); // inX.txt, gdzie X
         to kolejny iterator
		filenameInput += ".txt"; // inX.txt, gdzie X to kolejny iterator

		ifstream fileInput;
		fileInput.open(filenameInput.c_str(), ios::in);

		//---WARUNEK STOPU---//
		if(!fileInput.is_open()){ // jeżeli plik nie istnieje to zakoñcz
            return 0;
		}

		//---CZYTANIE Z PLIKU---//
        fileInput>>PESEL;
        fileInput.close();

        Pesel.PeselValidator(PESEL); // funkcja uruchamiająca walidator -> pesel.cpp

		//---TWORZENIE DANYCH WYJŚCIOWYCH---//
        string filenameOutput = "out/out" + ss.str();
		filenameOutput += ".txt";
		ofstream fileOutput;
		fileOutput.open(filenameOutput.c_str(), ios::out);


        if (Pesel.valid == 1) {
            fileOutput<<"Numer PESEL jest prawidlowy\n";
            fileOutput<<"Rok urodzenia: "<<Pesel.getBirthYear()<<"\n";
            fileOutput<<"Miesiac urodzenia: "<<Pesel.getBirthMonth()<<"\n";
            fileOutput<<"Dzien urodzenia: "<<Pesel.getBirthDay()<<"\n";
            fileOutput<<"Plec: "<<Pesel.getSex()<<"\n";
        }
        else  {
            fileOutput<<"Numer PESEL jest nieprawidlowy\n";
        }
        i++; // zwiększa iterator, outX.txt
    }while(true);

    return 0;
}

	\end{verbatim}
	\subsubsection*{pesel.h}
	\begin{verbatim}
#ifndef PESEL_H
#define PESEL_H

#include <iostream>
#include <cstdlib>
#include <string>
#include <fstream>
#include <sstream>
#include <windows.h>

using namespace std;

class pesel{
    public:

    pesel();

    short PESEL[11];
    bool valid = 0;

    int getBirthYear();
    int getBirthMonth();
    int getBirthDay();
    const char* getSex();
    int checkSum();
    int checkMonth();
    int checkDay();
    int leapYear(int);
    void PeselValidator(char*);

    virtual ~pesel();
};

#endif // PESEL_H

	\end{verbatim}
	
	\subsubsection*{pesel.cpp}
	\begin{verbatim}

#include "pesel.h"

pesel::pesel()
{
    //ctor
}

pesel::~pesel()
{
    //dtor
}

//https://pl.wikipedia.org/wiki/PESEL

int pesel::getBirthYear(){
	/*Numeryczny zapis daty urodzenia przedstawiony jest w następującym porządku: dwie 
	ostatnie cyfry roku, miesiąc i dzień. 
	Dla odróżnienia poszczególnych stuleci przyjęto następującą metodę kodowania:

    dla osób urodzonych w latach 1900 do 1999 – miesiąc zapisywany jest w sposób 
    naturalny, tzn. dwucyfrowo od 01 do 12
    dla osób urodzonych w innych latach niż 1900–1999 dodawane są do numeru miesiąca 
    następujące wielkości:
        dla lat 1800–1899 – 80
        dla lat 2000–2099 – 20
        dla lat 2100–2199 – 40
        dla lat 2200–2299 – 60.*/
		
    int year;
    int month;

    year = 10 * PESEL[0];
    year += PESEL[1];
    month = 10 * PESEL[2];
    month += PESEL[3];

    if (month > 80 && month < 93){
        year += 1800;
    }
    else
        if (month > 0 && month < 13){
            year += 1900;
        }
    else
        if (month > 20 && month < 33){
            year += 2000;
        }
    else
        if (month > 40 && month < 53){
            year += 2100;
        }
    else
        if (month > 60 && month < 73){
            year += 2200;
        }
    return year;
}

int pesel::getBirthMonth(){
    int month;
    month = 10 * PESEL[2];
    month += PESEL[3];

    if (month > 80 && month < 93){
        month -= 80;
    }
    else
        if (month > 20 && month < 33){
            month -= 20;
        }
    else
        if (month > 40 && month < 53){
            month -= 40;
        }
    else
        if (month > 60 && month < 73){
            month -= 60;
        }
    return month;
}

int pesel::getBirthDay(){
    int day;

    day = 10 * PESEL[4];
    day += PESEL[5];

    return day;
}

const char* pesel::getSex(){
	/*Informacja o płci osoby, której zestaw informacji jest identyfikowany, zawarta
	 jest na 10. (przedostatniej) pozycji numeru PESEL.

    cyfry 0, 2, 4, 6, 8 – oznaczają płeć żeńską
    cyfry 1, 3, 5, 7, 9 – oznaczają płeć męską

	Po zmianie płci przydzielany jest nowy numer PESEL*/
	
    if (valid){
        if (PESEL[9] % 2 == 1) {
            return "Mezczyzna";
        }
    else{
        return "Kobieta";
    }
	}
	else{
		return "-";
	}
}

int pesel::checkSum(){
	/*Jedenasta cyfra jest cyfrą kontrolną, służącą do wychwytywania przekłamań numeru.
	 Jest ona generowana na podstawie pierwszych dziesięciu cyfr. 
	Aby sprawdzić czy dany numer PESEL jest prawidłowy, należy, zakładając, że litery a-j
	 to kolejne cyfry numeru od lewej, obliczyć wyrażenie:
	 
	9×a + 7×b + 3×c + 1×d + 9×e + 7×f + 3×g + 1×h + 9×i + 7×j
	Jeżeli ostatnia cyfra otrzymanego wyniku nie jest równa cyfrze kontrolnej, to znaczy,
	 że numer zawiera błąd*/
	
    int sum = 9 * PESEL[0] +
    7 * PESEL[1] +
    3 * PESEL[2] +
    1 * PESEL[3] +
    9 * PESEL[4] +
    7 * PESEL[5] +
    3 * PESEL[6] +
    1 * PESEL[7] +
    9 * PESEL[8] +
    7 * PESEL[9];
    sum  = sum % 10;

    if (sum == PESEL[10]) return 1;
    else  return 0;
}

int pesel::checkMonth(){
    int month = getBirthMonth();

    if (month > 0 && month < 13){
        return 1;
    }
    else{
        return 0;
    }
}

int pesel::checkDay(){
    int year = getBirthYear();
    int month = getBirthMonth();
    int day = getBirthDay();
    if ((day > 0 && day < 32) &&
        (month == 1 || month == 3 || month == 5 ||
        month == 7 || month == 8 || month == 10 ||
        month == 12)) {
        return 1;
    }
    else
        if ((day > 0 && day < 31) &&
            (month == 4 || month == 6 || month == 9 ||
            month == 11)) {
            return 1;
        }
    else
        if ((day > 0 && day < 30 && leapYear(year)) || // warunk co do roku 
        przestępnego
            (day > 0 && day < 29 && !leapYear(year))) {
            return 1;
        }
    else{
        return 0;
    }
}

int pesel::leapYear(int year){
    if (year % 4 == 0 && year % 100 != 0 || year % 400 == 0)
        return 1;
    else
        return 0;
}

void pesel::PeselValidator(char *PESELNumber){
    int i;

    if (strlen(PESELNumber) != 11){
        valid = 0;
    }
    else{
        for (i = 0; i < 11; i++){
            PESEL[i] = PESELNumber[i] - 48;		//char to int, wg. tablicy ASCII
             aby tego dokonoać odejmujemy 48
        }
    if (checkSum() && checkMonth() && checkDay()){
        valid = 1;
    }
    else{
        valid = 0;
    }
	}
}

	
	\end{verbatim}
	
	\subsection*{skrypt\char`_python.py}
	
	\begin{verbatim}
	
#!/usr/bin/python
# -*- coding: utf-8 -*-
import os, sys
import time  # timestamp
import datetime  # timestamp
import webbrowser  # przegladarka
import os  # pliki
import shutil  # foldery

for file in os.listdir("backup"):  # wszystkie pliki w folderze backup
    if file.endswith(".html"):  # jesli plik jest rozszerzenia html
        lastBackup = file  # to przypisz go do zmiennej lastBackup, ostatni plik 
        to ostatni backup
        backupExists = True  # ustaw flage, ze backup istnieje

try:  # sprawdz czy zmienna lastBackup istnieje
    lastBackup
except NameError:  # jesli nie
    backupExists = False  # ustaw flage, ze to bedzie pierwszy backup

outFiles = []  # pusta lista przechowujaca wszystkie indeksy

for file in os.listdir("out"):  # wszystkie pliki w folderze out
    if file.startswith("out") and file.endswith(".txt"):  # jesli plik jest o
     nazwie out*.txt
        outFiles.append(file)  # to dopisz go do listy
        lastOut = True  # ustaw flage, ze ostatni plik wyjscia istnieje

for i in range(len(outFiles)):  # dla wszystkich plikow out nadpisz nazwe indeksami
    outFiles[i] = int(outFiles[i][3:len(outFiles[i])-4])  # outx.txt - usuwamy 3
     pierwsze i 4 ostatnie, rzutowanie int

if not len(outFiles):  # jesli lista plikow wyjsciowych jest pusta
    lastOut = False  # ustaw flage, ze zaden plik wyjsciowy nie istnieje

if lastOut:  # jesli plik wyjsciowy istnieje
    index = max(outFiles)  # to ustaw index jako maksymalny

ts = time.time()  # timestamp
st = datetime.datetime.fromtimestamp(ts).strftime("%Y-%m-%d_%H-%M-%S")  # konwersja 
timestampu na date

fileOut = open("backup/" + st + ".html", "a+")  # stworz nowy plik backup z data

if backupExists:  # jesli jakis backup istnieje
    fileTemp = open("backup/" + lastBackup).read()  # skopiuj zawartosc poprzedniego 
    backupu do zmiennej tymczasowej
    shutil.copyfile("backup/" + lastBackup, "backup/" + st + ".html")  # skopiuj 
    ostatni backup do nowego pliku
    while index:  # poki dzialamy na indeksach (od ostatniego indeksu wyjscia do 1)
        iTemp = "<tr id=" + str(index)  # string oznaczajacy wpis wiersza w HTML 
        o podanym indeksie (np. dla out3 to 3)
        if iTemp in fileTemp:  # jesli znalazl
            indexTemp = fileTemp.index(iTemp) + 7  # pomin <tr id= i przejdz do liczby
            i = int(fileTemp[indexTemp:indexTemp + len(str(index))]) + 1  # index + 1, bo taki index juz mamy
            break
        index -= 1  # a jesli nie znalazl to zmniejsza index (czyli sprawdza out2, 
        wedlug przykladu)
    del fileTemp  # zwolnij pamiec, tej zmiennej nie bedziemy juz uzywac
    del iTemp  # zwolnij pamiec, tej zmiennej nie bedziemy juz uzywac
    del indexTemp  # zwolnij pamiec, tej zmiennej nie bedziemy juz uzywac
    del index  # zwolnij pamiec, tej zmiennej nie bedziemy juz uzywac

else:  # jesli jednak backup nie istnieje
    fileOut.write("<html><head><meta charset='utf-8' />"
				  "<link rel='Stylesheet' href='../css/style.css'><script>var url = window.location.
				  pathname; "
                  "var file = url.substring(url.lastIndexOf('/')+1);"
                  "filename = file.substr(0, file.length - 5);"
                  "document.title = 'Backup ' + filename;</script><meta charset='UTF-8'
                  >"
                  "</head><body><h1>"
                  "<script>document.write(document.title)</script></h1>"
				  "<h2>Dawid Bitner IA</h2>"
				  "<h3>parser PESEL - jezyki skryptowe - projekt</h3>"
                  "<table>"
                  "<tr><th>Input</th><th>Output</th></tr>")  # to stworz nowy plik HTML
    i = 1  # i ustaw zmienna indeksu na 1

while True:
    try:  # sprawdz czy sie pliki otworza
        fileInput = open("in/in" + str(i) + ".txt", "r")
        fileOutput = open("out/out" + str(i) + ".txt", "r")
    except FileNotFoundError as e:  # warunek stopu, jesli pliki sie nie otworza to 
    konczy program
        break

    fileOut.write("<tr id=" + str(i) + "><td style='width:10%; text-align:center'>" 
    + fileInput.readline() + "</td><td>"
                  + fileOutput.read()+"</td></tr>")  # wpisz zawartosc plikow in i out do tabeli

    i += 1  # powieksz iterator
    fileInput.close()  # zamknij plik wejsciowy
    fileOutput.close()  # zamknij plik wyjsciowy

fileOut.close()  # jesli warunek stopu zostanie spelniony (wyjdzie z petli) to zamknij 
plik backup
webbrowser.open("file://" + os.path.realpath(fileOut.name))  # otworz przegladarke
	
	
	\end{verbatim}
\end{document}
